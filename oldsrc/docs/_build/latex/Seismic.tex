%% Generated by Sphinx.
\def\sphinxdocclass{report}
\documentclass[letterpaper,10pt,brazil]{sphinxmanual}
\ifdefined\pdfpxdimen
   \let\sphinxpxdimen\pdfpxdimen\else\newdimen\sphinxpxdimen
\fi \sphinxpxdimen=.75bp\relax

\usepackage[utf8]{inputenc}
\ifdefined\DeclareUnicodeCharacter
 \ifdefined\DeclareUnicodeCharacterAsOptional
  \DeclareUnicodeCharacter{"00A0}{\nobreakspace}
  \DeclareUnicodeCharacter{"2500}{\sphinxunichar{2500}}
  \DeclareUnicodeCharacter{"2502}{\sphinxunichar{2502}}
  \DeclareUnicodeCharacter{"2514}{\sphinxunichar{2514}}
  \DeclareUnicodeCharacter{"251C}{\sphinxunichar{251C}}
  \DeclareUnicodeCharacter{"2572}{\textbackslash}
 \else
  \DeclareUnicodeCharacter{00A0}{\nobreakspace}
  \DeclareUnicodeCharacter{2500}{\sphinxunichar{2500}}
  \DeclareUnicodeCharacter{2502}{\sphinxunichar{2502}}
  \DeclareUnicodeCharacter{2514}{\sphinxunichar{2514}}
  \DeclareUnicodeCharacter{251C}{\sphinxunichar{251C}}
  \DeclareUnicodeCharacter{2572}{\textbackslash}
 \fi
\fi
\usepackage{cmap}
\usepackage[T1]{fontenc}
\usepackage{amsmath,amssymb,amstext}
\usepackage{babel}
\usepackage{times}
\usepackage[Sonny]{fncychap}
\usepackage[dontkeepoldnames]{sphinx}

\usepackage{geometry}

% Include hyperref last.
\usepackage{hyperref}
% Fix anchor placement for figures with captions.
\usepackage{hypcap}% it must be loaded after hyperref.
% Set up styles of URL: it should be placed after hyperref.
\urlstyle{same}
\addto\captionsbrazil{\renewcommand{\contentsname}{Contents:}}

\addto\captionsbrazil{\renewcommand{\figurename}{Fig.}}
\addto\captionsbrazil{\renewcommand{\tablename}{Tabela}}
\addto\captionsbrazil{\renewcommand{\literalblockname}{Listagem}}

\addto\captionsbrazil{\renewcommand{\literalblockcontinuedname}{continuação da página anterior}}
\addto\captionsbrazil{\renewcommand{\literalblockcontinuesname}{continues on next page}}

\addto\extrasbrazil{\def\pageautorefname{página}}

\setcounter{tocdepth}{1}



\title{Seismic Documentation}
\date{dez 18, 2017}
\release{1}
\author{Igor Morgado}
\newcommand{\sphinxlogo}{\vbox{}}
\renewcommand{\releasename}{Release}
\makeindex

\begin{document}

\maketitle
\sphinxtableofcontents
\phantomsection\label{\detokenize{index::doc}}



\chapter{README}
\label{\detokenize{readme:readme}}\label{\detokenize{readme:welcome-to-seismic-s-documentation}}\label{\detokenize{readme::doc}}
A readme


\chapter{CHANGELOG}
\label{\detokenize{changelog::doc}}\label{\detokenize{changelog:changelog}}
A changlog


\chapter{Seismic}
\label{\detokenize{source/modules::doc}}\label{\detokenize{source/modules:seismic}}

\section{Base functions}
\label{\detokenize{source/base:base-functions}}\label{\detokenize{source/base::doc}}

\subsection{Module contents}
\label{\detokenize{source/base:module-base}}\label{\detokenize{source/base:module-contents}}\index{base (módulo)}
This code implements all algorithms described on
SILVA2015 - Introducao ao metodo sismico: Modelagem Computacional

Serao utilizadas a principio as mesmas variaves e nomeclaturas usadas
no artigo
\index{Vp() (no módulo base)}

\begin{fulllineitems}
\phantomsection\label{\detokenize{source/base:base.Vp}}\pysiglinewithargsret{\sphinxcode{base.}\sphinxbfcode{Vp}}{\emph{density=None}, \emph{young=None}, \emph{poisson=None}, \emph{shear=None}, \emph{deformation=None}, \emph{distance=None}, \emph{time=None}}{}
Retorna a velocidade da onda compressional (P) pelas propriedades dadas.

Para computacao, deve se entrar uma das seguintes combinacoes de
parametros:
* density, young, poisson;
* density, shear, deformation;
* distance, time
\begin{quote}\begin{description}
\item[{Parâmetros}] \leavevmode\begin{itemize}
\item {} 
\sphinxstyleliteralstrong{density} (\sphinxstyleliteralemphasis{float}) \textendash{} Densidade do material (rho).

\item {} 
\sphinxstyleliteralstrong{young} (\sphinxstyleliteralemphasis{float}) \textendash{} Coeficiente de Young (E).

\item {} 
\sphinxstyleliteralstrong{poisson} (\sphinxstyleliteralemphasis{float}) \textendash{} Coeficiente de poisson (sigma)

\item {} 
\sphinxstyleliteralstrong{shear} (\sphinxstyleliteralemphasis{float}) \textendash{} Coeficiente compressional (mu)

\item {} 
\sphinxstyleliteralstrong{deformation} (\sphinxstyleliteralemphasis{float}) \textendash{} Deformacao do material (k) (Bulk’s)

\item {} 
\sphinxstyleliteralstrong{distance} (\sphinxstyleliteralemphasis{float}) \textendash{} Distancia em metros (x)

\item {} 
\sphinxstyleliteralstrong{time} (\sphinxstyleliteralemphasis{float}) \textendash{} tempo percorrido (t).

\end{itemize}

\item[{Retorna}] \leavevmode
Velocidade da onda P.

\item[{Tipo de retorno}] \leavevmode
float

\end{description}\end{quote}

\end{fulllineitems}

\index{Vs() (no módulo base)}

\begin{fulllineitems}
\phantomsection\label{\detokenize{source/base:base.Vs}}\pysiglinewithargsret{\sphinxcode{base.}\sphinxbfcode{Vs}}{\emph{density}, \emph{shear=None}, \emph{young=None}, \emph{poisson=None}}{}
Returns shear velocity based on physical coefficients.

You should supply one of the two parameter combinations:
\begin{itemize}
\item {} 
density and shear;

\item {} 
density, Young and Poisson coefficient

\end{itemize}

This function verifies the values, therefore is slower, if you need faster
function, call directly:
\begin{itemize}
\item {} 
velocity\_s\_by\_shear

\item {} 
velocity\_s\_by\_young

\end{itemize}
\begin{quote}\begin{description}
\item[{Parâmetros}] \leavevmode\begin{itemize}
\item {} 
\sphinxstyleliteralstrong{density} (\sphinxstyleliteralemphasis{float}) \textendash{} Density (non negative) ( \(\rho\) ).

\item {} 
\sphinxstyleliteralstrong{shear} (\sphinxstyleliteralemphasis{float}) \textendash{} Shear (non negative) ( \(\mu\) ).

\item {} 
\sphinxstyleliteralstrong{young} (\sphinxstyleliteralemphasis{float}) \textendash{} Young coefficient (E).

\item {} 
\sphinxstyleliteralstrong{poisson} (\sphinxstyleliteralemphasis{float}) \textendash{} Poisson coefficient. \(0 < \sigma < 1/2\) .

\end{itemize}

\item[{Retorna}] \leavevmode
Shear velocity

\item[{Tipo de retorno}] \leavevmode
float

\end{description}\end{quote}

\end{fulllineitems}

\index{acoustic\_impedance() (no módulo base)}

\begin{fulllineitems}
\phantomsection\label{\detokenize{source/base:base.acoustic_impedance}}\pysiglinewithargsret{\sphinxcode{base.}\sphinxbfcode{acoustic\_impedance}}{\emph{density}, \emph{velocity}}{}
Returns acoustic impedance
\begin{quote}\begin{description}
\item[{Parâmetros}] \leavevmode\begin{itemize}
\item {} 
\sphinxstyleliteralstrong{density} (\sphinxstyleliteralemphasis{float}) \textendash{} Density (rho).

\item {} 
\sphinxstyleliteralstrong{velocity} (\sphinxstyleliteralemphasis{float}) \textendash{} Wave velocity (V)

\end{itemize}

\item[{Retorna}] \leavevmode
Acoustic impedance

\item[{Tipo de retorno}] \leavevmode
float

\end{description}\end{quote}

\end{fulllineitems}

\index{frequency() (no módulo base)}

\begin{fulllineitems}
\phantomsection\label{\detokenize{source/base:base.frequency}}\pysiglinewithargsret{\sphinxcode{base.}\sphinxbfcode{frequency}}{\emph{initial\_time}, \emph{end\_time}}{}
Retorna a frequencia baseado no intervalo de tempo.
\begin{quote}\begin{description}
\item[{Parâmetros}] \leavevmode\begin{itemize}
\item {} 
\sphinxstyleliteralstrong{t0} (\sphinxstyleliteralemphasis{float}) \textendash{} Inicio do intervalo.

\item {} 
\sphinxstyleliteralstrong{t1} (\sphinxstyleliteralemphasis{float}) \textendash{} Fim do intervalo.

\end{itemize}

\item[{Retorna}] \leavevmode
Frequencia.

\item[{Tipo de retorno}] \leavevmode
float

\end{description}\end{quote}

\end{fulllineitems}

\index{period() (no módulo base)}

\begin{fulllineitems}
\phantomsection\label{\detokenize{source/base:base.period}}\pysiglinewithargsret{\sphinxcode{base.}\sphinxbfcode{period}}{\emph{initial\_time}, \emph{final\_time}}{}
Retorna o periodo do intervalo.
\begin{quote}\begin{description}
\item[{Parâmetros}] \leavevmode\begin{itemize}
\item {} 
\sphinxstyleliteralstrong{initial\_time} (\sphinxstyleliteralemphasis{float}) \textendash{} Inicio do intervalo.

\item {} 
\sphinxstyleliteralstrong{final\_time} (\sphinxstyleliteralemphasis{float}) \textendash{} Fim do intervalo.

\end{itemize}

\item[{Retorna}] \leavevmode
Tamanho do intervalo.

\item[{Tipo de retorno}] \leavevmode
float

\end{description}\end{quote}

\end{fulllineitems}

\index{reflection\_coefficient() (no módulo base)}

\begin{fulllineitems}
\phantomsection\label{\detokenize{source/base:base.reflection_coefficient}}\pysiglinewithargsret{\sphinxcode{base.}\sphinxbfcode{reflection\_coefficient}}{\emph{z1}, \emph{z2}}{}
Returns the reflection coefficient.
\begin{quote}\begin{description}
\item[{Parâmetros}] \leavevmode\begin{itemize}
\item {} 
\sphinxstyleliteralstrong{z1} (\sphinxstyleliteralemphasis{float}) \textendash{} Acoustic impedance of upper layer.

\item {} 
\sphinxstyleliteralstrong{z2} (\sphinxstyleliteralemphasis{float}) \textendash{} Acoustic impedance of lower layer.

\end{itemize}

\item[{Retorna}] \leavevmode
Reflection coefficient.

\item[{Tipo de retorno}] \leavevmode
float

\end{description}\end{quote}

\end{fulllineitems}

\index{velocity\_p\_by\_distance() (no módulo base)}

\begin{fulllineitems}
\phantomsection\label{\detokenize{source/base:base.velocity_p_by_distance}}\pysiglinewithargsret{\sphinxcode{base.}\sphinxbfcode{velocity\_p\_by\_distance}}{\emph{x}, \emph{t}}{}
Retorna a velocidade da onda compressional dados distancia e tempo
\begin{quote}\begin{description}
\item[{Parâmetros}] \leavevmode\begin{itemize}
\item {} 
\sphinxstyleliteralstrong{x} (\sphinxstyleliteralemphasis{float}) \textendash{} Distancia percorrida.

\item {} 
\sphinxstyleliteralstrong{t} (\sphinxstyleliteralemphasis{float}) \textendash{} Tempo percorrido.

\end{itemize}

\item[{Retorna}] \leavevmode
velocidade compressional

\item[{Tipo de retorno}] \leavevmode
float

\end{description}\end{quote}

\end{fulllineitems}

\index{velocity\_p\_by\_shear() (no módulo base)}

\begin{fulllineitems}
\phantomsection\label{\detokenize{source/base:base.velocity_p_by_shear}}\pysiglinewithargsret{\sphinxcode{base.}\sphinxbfcode{velocity\_p\_by\_shear}}{\emph{rho}, \emph{mu}, \emph{k}}{}
Retorna a velocidade da onda P (compressional) pelas propriedades da
rocha.
\begin{quote}\begin{description}
\item[{Parâmetros}] \leavevmode\begin{itemize}
\item {} 
\sphinxstyleliteralstrong{rho} (\sphinxstyleliteralemphasis{float}) \textendash{} Densidade do material.

\item {} 
\sphinxstyleliteralstrong{mu} (\sphinxstyleliteralemphasis{float}) \textendash{} Modulo de cisalhamento (mede deformacao sem mudanca de

\item {} 
\sphinxstyleliteralstrong{volume}\sphinxstyleliteralstrong{)}\sphinxstyleliteralstrong{} \textendash{} 

\item {} 
\sphinxstyleliteralstrong{k} (\sphinxstyleliteralemphasis{float}) \textendash{} Modulo da deformacao volumetrica (Bulk).

\end{itemize}

\item[{Retorna}] \leavevmode
velocidade compressional.

\item[{Tipo de retorno}] \leavevmode
float

\end{description}\end{quote}

\end{fulllineitems}

\index{velocity\_p\_by\_young() (no módulo base)}

\begin{fulllineitems}
\phantomsection\label{\detokenize{source/base:base.velocity_p_by_young}}\pysiglinewithargsret{\sphinxcode{base.}\sphinxbfcode{velocity\_p\_by\_young}}{\emph{rho}, \emph{E}, \emph{sigma}}{}
Retorna a velocidade da onda P (compressional) pelas propriedades da
rocha.
\begin{quote}\begin{description}
\item[{Parâmetros}] \leavevmode\begin{itemize}
\item {} 
\sphinxstyleliteralstrong{rho} (\sphinxstyleliteralemphasis{float}) \textendash{} densidade do material.

\item {} 
\sphinxstyleliteralstrong{E} (\sphinxstyleliteralemphasis{float}) \textendash{} Modulo de Young (proporcionalidade entre a tensao e a

\item {} 
\sphinxstyleliteralstrong{deformacao}\sphinxstyleliteralstrong{)}\sphinxstyleliteralstrong{} \textendash{} 

\item {} 
\sphinxstyleliteralstrong{sigma} (\sphinxstyleliteralemphasis{float}) \textendash{} Coeficiente de Poisson (razao da contracao transversal

\item {} 
\sphinxstyleliteralstrong{extensao longitudinal.} (\sphinxstyleliteralemphasis{e}) \textendash{} 

\end{itemize}

\item[{Retorna}] \leavevmode
Velocidade compressional.

\item[{Tipo de retorno}] \leavevmode
float

\end{description}\end{quote}

\end{fulllineitems}

\index{velocity\_ratio() (no módulo base)}

\begin{fulllineitems}
\phantomsection\label{\detokenize{source/base:base.velocity_ratio}}\pysiglinewithargsret{\sphinxcode{base.}\sphinxbfcode{velocity\_ratio}}{\emph{deformation=None}, \emph{shear=None}, \emph{poisson=None}}{}
Returns velocity ratio P/S

The function requires “deformation and shear”  OR poisson coefficients. If
all three are supplied Poisson is ignored.

This function is slower since it does verifications but safer.  If you
want to call without value verification use:
* velocity\_ratio\_by\_deformation
* velocity\_ratio\_by\_poisson
\begin{quote}\begin{description}
\item[{Parâmetros}] \leavevmode\begin{itemize}
\item {} 
\sphinxstyleliteralstrong{deformation} (\sphinxstyleliteralemphasis{float}) \textendash{} Deformation factor (k), must be positive.

\item {} 
\sphinxstyleliteralstrong{shear} (\sphinxstyleliteralemphasis{float}) \textendash{} Shear factor (mu), must be positive.

\item {} 
\sphinxstyleliteralstrong{poisson} (\sphinxstyleliteralemphasis{float}) \textendash{} Poisson coeficient (sigma),  0 \textless{} sigma \textless{} 1/2.

\end{itemize}

\item[{Retorna}] \leavevmode
Velocity ratio P/S. Always a number bigger than 1.

\item[{Tipo de retorno}] \leavevmode
float

\end{description}\end{quote}

\end{fulllineitems}

\index{velocity\_ratio\_by\_deformation() (no módulo base)}

\begin{fulllineitems}
\phantomsection\label{\detokenize{source/base:base.velocity_ratio_by_deformation}}\pysiglinewithargsret{\sphinxcode{base.}\sphinxbfcode{velocity\_ratio\_by\_deformation}}{\emph{k}, \emph{mu}}{}
Retorna a razao entre a velocidade P e S.
\begin{quote}\begin{description}
\item[{Parâmetros}] \leavevmode\begin{itemize}
\item {} 
\sphinxstyleliteralstrong{deformation} (\sphinxstyleliteralemphasis{float}) \textendash{} Coeficiente de deformacao (de Bulk)

\item {} 
\sphinxstyleliteralstrong{shear} (\sphinxstyleliteralemphasis{float}) \textendash{} Coeficiente de cisalhamento

\end{itemize}

\item[{Retorna}] \leavevmode
Razao da velocidade Vp/Vs.

\item[{Tipo de retorno}] \leavevmode
float

\end{description}\end{quote}

\end{fulllineitems}

\index{velocity\_ratio\_by\_poisson() (no módulo base)}

\begin{fulllineitems}
\phantomsection\label{\detokenize{source/base:base.velocity_ratio_by_poisson}}\pysiglinewithargsret{\sphinxcode{base.}\sphinxbfcode{velocity\_ratio\_by\_poisson}}{\emph{sigma}}{}
Retorna a razao entre a Vp e Vs
\begin{quote}\begin{description}
\item[{Parâmetros}] \leavevmode
\sphinxstyleliteralstrong{poisson} (\sphinxstyleliteralemphasis{float}) \textendash{} Coeficiente de poisson

\item[{Retorna}] \leavevmode
Razao da velocidade Vp/Vs

\item[{Tipo de retorno}] \leavevmode
float

\end{description}\end{quote}

\end{fulllineitems}

\index{velocity\_s\_by\_shear() (no módulo base)}

\begin{fulllineitems}
\phantomsection\label{\detokenize{source/base:base.velocity_s_by_shear}}\pysiglinewithargsret{\sphinxcode{base.}\sphinxbfcode{velocity\_s\_by\_shear}}{\emph{rho}, \emph{mu}}{}
Retorna a velocidade da onda S (cisalhante) dados mu e rho.
\begin{quote}\begin{description}
\item[{Parâmetros}] \leavevmode\begin{itemize}
\item {} 
\sphinxstyleliteralstrong{rho} (\sphinxstyleliteralemphasis{float}) \textendash{} Coeficiente de Cisalhamento

\item {} 
\sphinxstyleliteralstrong{mu} (\sphinxstyleliteralemphasis{float}) \textendash{} Coeficiente de densidade.

\end{itemize}

\item[{Retorna}] \leavevmode
Velocidade cisalhante

\item[{Tipo de retorno}] \leavevmode
float

\end{description}\end{quote}

\end{fulllineitems}

\index{velocity\_s\_by\_young() (no módulo base)}

\begin{fulllineitems}
\phantomsection\label{\detokenize{source/base:base.velocity_s_by_young}}\pysiglinewithargsret{\sphinxcode{base.}\sphinxbfcode{velocity\_s\_by\_young}}{\emph{rho}, \emph{E}, \emph{sigma}}{}
Retorna a velocidade da onda S dados sigma e rho.
\begin{quote}\begin{description}
\item[{Parâmetros}] \leavevmode\begin{itemize}
\item {} 
\sphinxstyleliteralstrong{rho} (\sphinxstyleliteralemphasis{float}) \textendash{} Coeficiente de densidade (rho)

\item {} 
\sphinxstyleliteralstrong{E} (\sphinxstyleliteralemphasis{float}) \textendash{} Young’s soefficient

\item {} 
\sphinxstyleliteralstrong{sigma} (\sphinxstyleliteralemphasis{float}) \textendash{} Coeficiente de Poisson (sigma)

\end{itemize}

\item[{Retorna}] \leavevmode
Velocidade cisalhante

\item[{Tipo de retorno}] \leavevmode
float

\end{description}\end{quote}

\end{fulllineitems}

\index{wave\_velocity() (no módulo base)}

\begin{fulllineitems}
\phantomsection\label{\detokenize{source/base:base.wave_velocity}}\pysiglinewithargsret{\sphinxcode{base.}\sphinxbfcode{wave\_velocity}}{\emph{initial\_position}, \emph{initial\_time}, \emph{final\_position}, \emph{final\_time}}{}
Retorna a velocidade da onda como a razao entre duas cristas de onda.
\begin{quote}\begin{description}
\item[{Parâmetros}] \leavevmode\begin{itemize}
\item {} 
\sphinxstyleliteralstrong{initial\_position} (\sphinxstyleliteralemphasis{float}) \textendash{} Inicio do Intervalo.

\item {} 
\sphinxstyleliteralstrong{initial\_time} (\sphinxstyleliteralemphasis{float}) \textendash{} Inicio do intervalo.

\item {} 
\sphinxstyleliteralstrong{final\_position} (\sphinxstyleliteralemphasis{float}) \textendash{} Final do intervalo.

\item {} 
\sphinxstyleliteralstrong{final\_time} (\sphinxstyleliteralemphasis{float}) \textendash{} Fim do intervalo.

\end{itemize}

\item[{Retorna}] \leavevmode
Velocidade da onda

\item[{Tipo de retorno}] \leavevmode
float

\end{description}\end{quote}

\end{fulllineitems}

\index{wavelength() (no módulo base)}

\begin{fulllineitems}
\phantomsection\label{\detokenize{source/base:base.wavelength}}\pysiglinewithargsret{\sphinxcode{base.}\sphinxbfcode{wavelength}}{\emph{initial\_position}, \emph{final\_position}}{}
Retorna ‘lambda’ (comprimento de onda), dados dois picos r1, r0.
\begin{quote}\begin{description}
\item[{Parâmetros}] \leavevmode\begin{itemize}
\item {} 
\sphinxstyleliteralstrong{initial\_position} (\sphinxstyleliteralemphasis{float}) \textendash{} Inicio do Intervalo.

\item {} 
\sphinxstyleliteralstrong{final\_position} (\sphinxstyleliteralemphasis{float}) \textendash{} Final do intervalo.

\end{itemize}

\item[{Retorna}] \leavevmode
Comprimento de onda (lambda)

\item[{Tipo de retorno}] \leavevmode
float

\end{description}\end{quote}

\end{fulllineitems}



\section{Wave Propagation}
\label{\detokenize{source/wave::doc}}\label{\detokenize{source/wave:wave-propagation}}

\subsection{Module contents}
\label{\detokenize{source/wave:module-wave}}\label{\detokenize{source/wave:module-contents}}\index{wave (módulo)}\index{discrete\_wave\_step\_3d() (no módulo wave)}

\begin{fulllineitems}
\phantomsection\label{\detokenize{source/wave:wave.discrete_wave_step_3d}}\pysiglinewithargsret{\sphinxcode{wave.}\sphinxbfcode{discrete\_wave\_step\_3d}}{\emph{wave\_field}, \emph{velocity\_field}, \emph{dx}, \emph{dy}, \emph{dz}, \emph{dt}}{}
Executes the discrete wave timestep forward in time

\end{fulllineitems}

\index{fp\_second\_order() (no módulo wave)}

\begin{fulllineitems}
\phantomsection\label{\detokenize{source/wave:wave.fp_second_order}}\pysiglinewithargsret{\sphinxcode{wave.}\sphinxbfcode{fp\_second\_order}}{\emph{values}, \emph{interval}, \emph{axis=-1}}{}
Returns the second derivative of second order

values must have (at least) 3 points equally spaced.
\begin{quote}\begin{description}
\item[{Parâmetros}] \leavevmode\begin{itemize}
\item {} 
\sphinxstyleliteralstrong{values} (\sphinxstyleliteralemphasis{nparray}) \textendash{} Values of points in grid

\item {} 
\sphinxstyleliteralstrong{interval} (\sphinxstyleliteralemphasis{float}) \textendash{} Interval

\end{itemize}

\item[{Retorna}] \leavevmode
List of second order derivatives

\item[{Tipo de retorno}] \leavevmode
float

\end{description}\end{quote}

\end{fulllineitems}

\index{fp\_second\_order\_explicit\_terms() (no módulo wave)}

\begin{fulllineitems}
\phantomsection\label{\detokenize{source/wave:wave.fp_second_order_explicit_terms}}\pysiglinewithargsret{\sphinxcode{wave.}\sphinxbfcode{fp\_second\_order\_explicit\_terms}}{\emph{U}, \emph{axis=-1}}{}
Returns the terms of explicit second order first derivative

values must have (at least) 3 points equally spaced.
\begin{quote}\begin{description}
\item[{Parâmetros}] \leavevmode
\sphinxstyleliteralstrong{U} (\sphinxstyleliteralemphasis{nparray}) \textendash{} Values of points in grid

\item[{Retorna}] \leavevmode
List of second order derivatives

\item[{Tipo de retorno}] \leavevmode
float

\end{description}\end{quote}

\end{fulllineitems}

\index{fp\_second\_order\_term() (no módulo wave)}

\begin{fulllineitems}
\phantomsection\label{\detokenize{source/wave:wave.fp_second_order_term}}\pysiglinewithargsret{\sphinxcode{wave.}\sphinxbfcode{fp\_second\_order\_term}}{\emph{U}, \emph{axis=-1}}{}
Returns the second derivative of second order without the step

values must have (at least) 3 points equally spaced.
\begin{quote}\begin{description}
\item[{Parâmetros}] \leavevmode
\sphinxstyleliteralstrong{U} (\sphinxstyleliteralemphasis{nparray}) \textendash{} Value points in grid

\item[{Retorna}] \leavevmode
List of second order derivatives

\item[{Tipo de retorno}] \leavevmode
float

\end{description}\end{quote}

\end{fulllineitems}

\index{fpp\_fourth\_order() (no módulo wave)}

\begin{fulllineitems}
\phantomsection\label{\detokenize{source/wave:wave.fpp_fourth_order}}\pysiglinewithargsret{\sphinxcode{wave.}\sphinxbfcode{fpp\_fourth\_order}}{\emph{U}, \emph{interval}, \emph{axis=-1}}{}
Returns the second derivative of fourth order

values must have (at least) 5 points equally spaced.
\begin{quote}\begin{description}
\item[{Parâmetros}] \leavevmode\begin{itemize}
\item {} 
\sphinxstyleliteralstrong{values} (\sphinxstyleliteralemphasis{nparray}) \textendash{} Values of points in grid

\item {} 
\sphinxstyleliteralstrong{interval} (\sphinxstyleliteralemphasis{float}) \textendash{} Interval

\end{itemize}

\item[{Retorna}] \leavevmode
List of fourth order derivatives

\item[{Tipo de retorno}] \leavevmode
float

\end{description}\end{quote}

\end{fulllineitems}

\index{fpp\_fourth\_order\_term() (no módulo wave)}

\begin{fulllineitems}
\phantomsection\label{\detokenize{source/wave:wave.fpp_fourth_order_term}}\pysiglinewithargsret{\sphinxcode{wave.}\sphinxbfcode{fpp\_fourth\_order\_term}}{\emph{U}, \emph{axis=-1}}{}
Returns the second derivative of fourth order term without the step

values must have (at least) 5 points equally spaced.
\begin{quote}\begin{description}
\item[{Parâmetros}] \leavevmode
\sphinxstyleliteralstrong{values} (\sphinxstyleliteralemphasis{nparray}) \textendash{} Values of points in grid

\item[{Retorna}] \leavevmode
List of fourth order derivatives

\item[{Tipo de retorno}] \leavevmode
float

\end{description}\end{quote}

\end{fulllineitems}



\section{Seismic Source}
\label{\detokenize{source/source::doc}}\label{\detokenize{source/source:seismic-source}}

\subsection{Module contents}
\label{\detokenize{source/source:module-source}}\label{\detokenize{source/source:module-contents}}\index{source (módulo)}\index{source() (no módulo source)}

\begin{fulllineitems}
\phantomsection\label{\detokenize{source/source:source.source}}\pysiglinewithargsret{\sphinxcode{source.}\sphinxbfcode{source}}{}{}
A seismic source

\end{fulllineitems}



\section{Stability Analysis}
\label{\detokenize{source/stability::doc}}\label{\detokenize{source/stability:stability-analysis}}

\subsection{Module contents}
\label{\detokenize{source/stability:module-stability}}\label{\detokenize{source/stability:module-contents}}\index{stability (módulo)}
This module will evaluate model stability
\index{dispersion() (no módulo stability)}

\begin{fulllineitems}
\phantomsection\label{\detokenize{source/stability:stability.dispersion}}\pysiglinewithargsret{\sphinxcode{stability.}\sphinxbfcode{dispersion}}{\emph{V}, \emph{dx}, \emph{dy}, \emph{dz}, \emph{f}, \emph{k=5}}{}
Evalutate the model dispersion.
\begin{quote}\begin{description}
\item[{Parâmetros}] \leavevmode\begin{itemize}
\item {} 
\sphinxstyleliteralstrong{V} (\sphinxstyleliteralemphasis{nparray}) \textendash{} is the velocity model

\item {} 
\sphinxstyleliteralstrong{dx} (\sphinxstyleliteralemphasis{float}) \textendash{} is the x interval

\item {} 
\sphinxstyleliteralstrong{dy} (\sphinxstyleliteralemphasis{float}) \textendash{} is the y interval

\item {} 
\sphinxstyleliteralstrong{dz} (\sphinxstyleliteralemphasis{float}) \textendash{} is the z interval

\item {} 
\sphinxstyleliteralstrong{f} (\sphinxstyleliteralemphasis{float}) \textendash{} is the cutdown frequency

\item {} 
\sphinxstyleliteralstrong{k} (\sphinxstyleliteralemphasis{float}) \textendash{} maximum number of samples in a wavelength.

\end{itemize}

\item[{Retorna}] \leavevmode
If the model is stable

\item[{Tipo de retorno}] \leavevmode
bool

\end{description}\end{quote}

\end{fulllineitems}

\index{stability() (no módulo stability)}

\begin{fulllineitems}
\phantomsection\label{\detokenize{source/stability:stability.stability}}\pysiglinewithargsret{\sphinxcode{stability.}\sphinxbfcode{stability}}{\emph{V}, \emph{dx}, \emph{dy}, \emph{dz}, \emph{dt}, \emph{mu=5}}{}
Evalutate the model stability.
\begin{quote}\begin{description}
\item[{Parâmetros}] \leavevmode\begin{itemize}
\item {} 
\sphinxstyleliteralstrong{V} (\sphinxstyleliteralemphasis{nparray}) \textendash{} is the velocity model

\item {} 
\sphinxstyleliteralstrong{dx} (\sphinxstyleliteralemphasis{float}) \textendash{} is the x interval

\item {} 
\sphinxstyleliteralstrong{dy} (\sphinxstyleliteralemphasis{float}) \textendash{} is the y interval

\item {} 
\sphinxstyleliteralstrong{dz} (\sphinxstyleliteralemphasis{float}) \textendash{} is the z interval

\item {} 
\sphinxstyleliteralstrong{dt} (\sphinxstyleliteralemphasis{float}) \textendash{} is the t interval (timestep)

\item {} 
\sphinxstyleliteralstrong{mu} (\sphinxstyleliteralemphasis{float}) \textendash{} is stability constant

\end{itemize}

\item[{Retorna}] \leavevmode
If the model is stable

\item[{Tipo de retorno}] \leavevmode
bool

\end{description}\end{quote}

\end{fulllineitems}



\chapter{Indices and tables}
\label{\detokenize{index:indices-and-tables}}\begin{itemize}
\item {} 
\DUrole{xref,std,std-ref}{genindex}

\item {} 
\DUrole{xref,std,std-ref}{modindex}

\item {} 
\DUrole{xref,std,std-ref}{search}

\end{itemize}


\renewcommand{\indexname}{Índice de Módulos Python}
\begin{sphinxtheindex}
\def\bigletter#1{{\Large\sffamily#1}\nopagebreak\vspace{1mm}}
\bigletter{b}
\item {\sphinxstyleindexentry{base}}\sphinxstyleindexpageref{source/base:\detokenize{module-base}}
\indexspace
\bigletter{s}
\item {\sphinxstyleindexentry{source}}\sphinxstyleindexpageref{source/source:\detokenize{module-source}}
\item {\sphinxstyleindexentry{stability}}\sphinxstyleindexpageref{source/stability:\detokenize{module-stability}}
\indexspace
\bigletter{w}
\item {\sphinxstyleindexentry{wave}}\sphinxstyleindexpageref{source/wave:\detokenize{module-wave}}
\end{sphinxtheindex}

\renewcommand{\indexname}{Índice}
\printindex
\end{document}